Fourthly, a comparison analysis between the mechanical properties of the studies soft materials and the human tendons involved in the knee motion is presented. Earlier it was mentioned the decision of focusing on the knee joint of the human lower limb, due to the large participation it has on the activities of daily living (ADLs). Finally, the last section of this chapter presents the summary of the relevant findings.


\section{Comparison Analysis}

One of the objective behind the characterization process previously described is to investigate the similarities between the viscoelastic properties of the selected soft materials and the human tendons involved in the motion of the knee joint. Therefore, the mechanical properties of the quadriceps tendon and the patellar tendon are extracted from the literature.

%Recently I found a second study by the same author I am referencing about the mechanical properties of the human behaviour. These are the references:

%\cite{schatzmann1998effect}
%\cite{staubli1999mechanical}

Finally, the mechanical properties of the human quadriceps tendon (Schatzmann, Brunner and St{\"a}ubli, 1998) and of the six soft materials are compiled and compared in ADDTABLE. In this table, the difference between the cross-sectional area of the human tendon and the materials is highlighted and reflected in the difference between their mechanical properties. However, even by matching the human tendon cross-sectional area, the current soft materials are not able to match its strength.

In the literature, stress relaxation experiments on the human patellar tendon have been performed (Johnson et al., 1994). The tests duration in this work is 300 seconds which differs from the three hours of our experimental setup. Therefore, the materials and the human tendon are compared on the time frame of 300 seconds. The achievable amount of stress relaxation is defined by Eq. (1) as follows:

Where achieved stress relaxation at the 300 seconds mark,  and  are the initial and final stress, respectively. 

Lastly, the data from the human patellar tendon is compared against the materials data in Table II.

In contrast with the great difference observed between the elastic properties of the soft materials and the human tendon (Table I), the viscoelastic properties of the soft  extracted from the stress relaxation tests are closer to the human tendon ones. These results are discussed in detail in Section 3.


{\huge The Best Candidate Explanation}

At this stage, the elastic and viscoelastic properties of the soft materials and the human tendon have been compared against each other. The results highlights that the Flourocarbon Rubber  has the most similarities to the human tendon mechanical properties. This is based on the two tables summarizing the elastic and viscoelastic properties of both.
To-do List
\begin{itemize}
    \item Define the criteria to choose which material is the most suitable
    \item Elastic parameters
    \begin{itemize}
        \item Stiffness at Toe region (small deformation)
        \item Stiffness at Elastic region (long deformation)
        \item Ultimate load
        \item Ultimate elongation
        \item Matching Factor(increment factor of the material cross-sectional area required to match the human tendon strength)
        
    \end{itemize}
    \item Viscoelastic parameters
    \begin{itemize}
        \item 
    \end{itemize}
\end{itemize}
