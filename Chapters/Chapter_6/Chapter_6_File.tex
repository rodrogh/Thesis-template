\chapter{Soft Series-elastic Actuator Modelling}

\section{Introduction}

This chapter focuses on assessing the mapping capabilities of neural networks when implemented in a control system. Moreover, the benefits of using a soft materials as the elastic element in a series-elastic actuator is assessed by comparing it against the traditional series-elastic actuator with a metallic spring.

\section{Traditional Series-elastic Actuator Model}

OBJECTIVE: Document the process in which the transfer function of the system is deducted, firstly for a traditional motor-load system, then for a traditional SEA and finally, for a soft SEA.

The process for modelling single robotic joint, based on an electric motor, is described in the book "Robotics: Control, Sensing, Vision, and Intelligence" from K.S. Fu. The dynamic behaviour of an electric motor can be defined by two constitutive equation, one for the electric part and other for the mechanical part of the motor. Following Kirchhoff voltage rule and Newton's second law we obtain the two following equations:

