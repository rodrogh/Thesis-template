\chapter{Introduction}
\section{ Background}

\Large{The background must give an overview of the current development, the progress of the research and the gaps in knowledge (research opportunities) }

The field of Soft robotics deals with the implementation of soft and deformable materials in traditional Robotics applications. The interest and research in this field has increased in the past decade. The latter is due to the outstanding developments in terms of manufacturing of soft materials, such as elastomers, and the development of new soft materials which can be stimulated, i.e. deformed, by heat, light, and magnetism. 

The coordination action called RoboSoft, supported by the IEEE Robotics and Automation Society (RAS) and the European Commission, has played a very important role in spreading the awareness of the wide number of Soft Robotics applications. The RoboSoft committee formally defines the field of Soft Robotics as ``Soft robot/devices that can actively interact with the environment and can undergo `large' deformations relying on inherent or structural compliance'' \cite{laschi2016soft}. The categorization of when a robotic application fall into the field of Soft Robotics depends on the material's Young Modulus, a mechanical property which relates the material's deformation with the amount of stress applied to it; which must be between the range of 10$^{2} - $10$^{6}$ Pascals (Pa). The Young's modulus is usually a measure of a solid material stiffness; a high value refers to a stiff material in the same way as a low value refers to an elastic or soft material. In the context of Soft Robotics, the term compliance it is most used instead of stiffness since it refers to the adaptability of the material under certain circumstances.

Early applications in Soft Robotics were inspired in nature, by observing that most biological organisms are not rigid, e.g. the human skeleton only contributes with 11\% of an adult's weight, on contrast, the skeletal muscle in our body only contributes with 42\% of the weight \cite{kim2013soft}. This inspiration gave birth to several soft bio-inspired robots, as well as the interest in studying the embodied intelligence of biological organisms. The latter refers to the ability of biological organisms to adapt to different situations by exploding their body morphology and properties. This is one of the main differences between soft and rigid robots. The embodied intelligence of soft robots release the controller from the task of accurately controlling the position of the robot and of constantly monitoring the working environment; allowing the controller to focus on the execution of commands. This is only possible with the implementation of soft and deformable materials able to automatically adapt to perturbations from the environment, such as uneven terrains and obstacles. Bio-inspired soft robots are now being developed for a broad range of applications, such as locomotion, manipulation, and even replicating biological processes such as digestion. The research done in the field of Soft Robotics has an interesting multidisciplinary potential. For example, the locomotion of caterpillars and snake could be study by building a soft robot which replicates this motion. The knowledge extracted from this could be useful for the development of actuated bendable soft cylinders which ultimately could replace the current rigid tools being used in laparoscopic surgery \cite{rus2015design}.

The mechanical behaviour of soft materials, is very difficult to model using traditional mathematical models, due to their non-linear, time-dependent and strain-rate-dependent stress response. This great challenge motivated the research in Soft Robotics to develop bio-inspired soft bodies, arms and legs able to perform the task at hand using minimal control. This caused a shift in the traditional design approach for rigid robots from ``rigidity by design, safety by sensors and control'' to design approach used in soft robots, ``safety by design, performance by control''. The added feature of safety, inherent in most soft robots, allowed the development of Physical Human-Robot Interaction (PHRI) applications \cite{filippini2008toward}. Many other challenges face by this emerging field are listed in \cite{laschi2016soft,trivedi2008soft} being: actuation of soft materials, development and implementation of soft sensors into soft robots, control systems able to deal with the nonlinear behaviour of soft materials, and modelling tools able to accurately predict the mechanical behaviour of soft materials in real-time. Most of these limitations come from the simple fact that soft robots cannot be considered as a chain of rigid links able to rotate or slide as common robots are, but soft robots are deformable and continuous which means that all the foundation in which Robotics is based on, is not easily transferable to the Soft Robotics field. 

The design and development of soft robots is very challenging, however, the potential benefits are many \cite{iida2011soft}. Foe example, soft robots have the potential of being very dexterous due to the ability of modifying their shape depending on the environment; soft robots can can manipulate objects of different shapes, sizes, and most importantly they are safe to interact with humans in the event of an unplanned collision. The latter benefits highlight the potential of Soft Robotics for PHRI applications such as, orthoses, surgical tools, and wearable devices. This research aims to contribute to the technological advance of Soft Robotics for human assistance applications.

For a long time, humans have pursued the idea of increasing their strength, stamina and speed through different means. Currently, we are relaying on technological advances to achieve the latter by building wearable robotic devices, commonly known as robotic exoskeletons. This wearable device was motivated by military applications where a soldier is required to carry a heavy load in its back for a long period of time, ultimately causing him injuries or early fatigue. The robotic exoskeleton is able to carry a payload and transmit the payload weight to the ground, ideally relieving the wearer from feeling the payload, which allows the wearer to walk greater distances without premature fatigue. The rigid nature of these devices and the big actuators implemented to achieve forces able to enhance humans, impose many limitations, such as, restriction of body movements, interference on subject natural biomechanics and high inertia which impedes the device to follow the subject intentions smoothly, creating a drag feeling \cite{asbeck2014stronger}.

In the field of Soft Robotics, research is being done to develop a soft version of a robotic exoskeleton, formally called soft exosuit, in an attempt to solve the previously mentioned limitations. Soft exosuits are wearable robotic systems meant to be worn in the same way as clothes, by attaching both the actuation and perception systems into a wearable structure, made of textiles, strapped to the human body which will ultimately assist the wearer's motions. Despite the many benefits of this technology, in comparison to the robotic exoskeleton, current developments of soft exosuits are not able to deliver enough mechanical power to be called an enhancement device. Currently, soft exosuits are considered an assistive device. This limitation is inherent from the idea of having a soft wearable device. The conversion of forces and torques, generated by the actuators attached to a wearable textile or directly to the user's body, into useful assistive torques for human joints assistance is very challenging. In a rigid exoskeleton the reaction forces produced by the actuators are sustained by the exoskeleton mechanical frame, but this is not the case in soft exosuits, where the reaction forces must be dissipated through the worn textile attached to the human body, which generates uncomfortable frictions between the user skin and the textile. Nevertheless, the human body naturally possesses areas able to sustain high amounts of forces, these areas are currently being exploited into soft exosuit designs to prevent discomfort, skin injury, and to increase the efficiency of transmitted forces. The latter principle is implemented in \cite{wehner2013lightweight} where a lower limb soft exosuit using pneumatic muscles is developed. The other, most common approach is to fix a wearable soft material to the skin, as in \cite{park2014design,park2011bio} where pneumatic artificial muscles (PAM) were attached to a soft cloth and disposed in such a way that they can mimic the biological musculoskeletal behaviour of a human foot.

Among the available soft exosuit developments, few of them implements a closed loop control system, due to the high complexity of dealing with the non-linearity of soft materials. On the other hand, several works focus on the characterization and testing of soft technologies as \cite{wehner2012exp} where an accurate characterization of a PAM is performed, which directly contribute to the development of previously mentioned soft exosuits. Many other technologies are being researched as well, such as, shape memory alloys (SMAs), which are metal alloys capable of changing its form when  properly stimulated. One of the main drawbacks of this technology is the amount of energy and time required to heat metal alloy and activate its shape shifting ability. The latter limitation is highlighted in \cite{Stirling2011}, where soft orthoses for foot and knee were developed SMA springs embedded in a soft fabric. The response of the device was to slow to be used for human assistance. Another commonly implemented actuation technology are cable-driven actuators, typically consisting of an electric motor and a Bowden cable. This electromechanical system create pulling forces by applying tension to a cable fixed in two different points. The cables must be routed along the body of the wearer to deliver the acting force to the joint of interest \cite{asbeck2013biologically,asbeck2015biologically}. The stroke length is not limited in this type of technology, as is the case for PAMs, since Bowden cables are flexible and can be routed in many different ways.

Many soft sensing technologies have also been developed, being hyper-elastic strain sensors the most common one, which are based on the change in resistance of a liquid metal alloy inside conducts embedded into an elastic polymer which allows the measurement of the body limb position and sustained forces; its performance allowed the development of a soft artificial skin \cite{park2012design}. This technology was later implemented in a soft exosuit to allow for the measurement of the human biomechanics. A comparison analysis between the latter approach and the well-established motion capture technology was performed in \cite{mengucc2014wearable}, where the implementation of strain sensors to capture the human biomechanics showed remarkable results.

The progress in the field of Soft Robotics has been very quick and very diverse. Most of the research is focused on studying the benefits of using a specific new soft material in a robotic application. Hence, most of the research leaves the control system in a second plane, mostly due to the complexity of developing a modelling tool to accurately predict the mechanical behaviour of soft materials. As previously mentioned, this is one of the main challenges in Soft Robotics applications and the main focus of this research.

In the literature, most of the work is based on one of two approaches, a model-driven or a data-driven approach. The model-driven, as suggested is based on using mathematical models to predict the mechanical behaviour of soft materials. Due to the time-dependent properties, or viscoelasticity, exhibited in most soft materials, model-driven approaches use a set of equations for model viscoelasticity known as the Linear Viscoelastic Models (LMVs). The models included in this group are: . These models describe a material using two basic mechanical components, a spring and a damper, arranged in different configurations, being the NAME the simplest model ... and NAME the most complex model of the LVMs. In practice, the NAME model can describe any viscoelastic materials as long as enough number of Maxwell Branches are included in the model.


Use this paragraph as an example:

At the time of writing this report, the exoskeleton technology has matured and is moving into series production. The nature of the exoskeleton as a complex mechatronic device poses many challenges. As it is described in later chapters, the exoskeleton must employ numerous distributed electronic devices, which need to be connected into a network. An efficient method is needed to reduce wiring complexity and weight. At the same time, a reliable and efficient solution is needed to deliver the amount of energy required by the hydraulics and on board electronics. The literature on exoskeletons and power systems provides limited information on this topic. However, it is worth noting that a communication system is an adequate solution along with a safe power system. This has been a well known problem which incentivized progress in the automotive and industrial automation sectors. The need to reduce the amount of wiring and complexity of distributed electronic systems led to the development of important communication technologies that are widely adopted standards nowadays. Finally, the increasing demand for energy and environmental concerns are driving forces toward renewable energy sources and wireless power transmission (WPT) technologies.

\section{Motivation}

\Large { The motivation represents the gaps in the knowledge attempted to be filled. These gaps must be highlighted by the literature review }

The global percentage of elderly population is constantly rising. The World Health Organization has estimated that the global percentage of people aged 65 or older will triple by 2050, with respect to 2010 \cite{Colombo2012}. Moreover, the amount of elderly people living alone is also increasing. Solely in the United Kingdom, there are 1 million people aged over 65 living on their own \cite{Hill2019}. Although, the social triggers of the mentioned phenomenons are out of the scope of this research, they represent a strong motivation to push forward the research on Soft Robotics applications for human assistance. Currently, there are viable concepts of assistive exosuits targeted to increase the quality of life of elderly people during activities of daily living, such as walking over ground, ascending stairs, using a chair. However, the concept of an assistive exosuit is relatively new, the first documented prototype is dated from 2013 \cite{wehner2013lightweight}. The idea behind an exosuit is to translate the proven concept of a robotic exoskeleton, which is a heavy and bulky wearable device aimed to enhance the strength of humans, into a soft, light weight and compliance version aimed to provide assistance to the elderly or disabled people. 

One of the main challenges in this field of research is the modelling of the mechanical behaviour of soft materials. The accuracy of current modelling approaches is restricted by the computational cost required. This prevent them to be deployed in most micro-controllers, due to the limited computational power available in this devices. Commercially ready exosuits might be possible in a decade time, meanwhile there is plenty of research to be done to translate well developed technologies used in Robotics into the field of Soft Robotics.

Summarizing, this research can improve the quality of life of the elderly and disabled population by developing a novel modelling approach which will enable current soft actuation technologies to be more reliable when being implemented in soft exosuits.

\section{Aims and Objectives}

\subsubsection{Aim}

The aims of this research are:

\begin{itemize}
    \item To investigate the concept of mimicking the viscoelastic mechanical properties of the human musculoskeletal system in Soft Robotics applications for human assistance of the lower limb.
    \item To identify and assess the most commonly used modelling approach for the prediction of viscoelastic stress response of soft materials.
    \item To assess the performance of the modelling approach when being implemented as part of a control system for the real-time prediction of the stress response of a material, in a simulation environment.
\end{itemize}

\subsubsection{Objectives}

In accordance to the research aims, the following objectives are identified:

\begin{itemize}
    \item Perform a literature review of the following topics:
    \begin{itemize}
        \item The biomechanics of the human the lower limb during activities of daily living.
        \item Soft Robotics applications for human assistance of the lower limb.
        \item Soft actuation technologies currently used for the assistance of the human knee joint.
        \item Modelling techniques being used for the prediction of the stress response of viscoelastic soft materials.
    \end{itemize}
    \item Compile a database for the human lower limb biomechanics during activities of daily living, such as: walking, ascending/descending stairs, ascending/descending ramps and sitting down/standing up from chair.
    \item Characterize the viscoelastic mechanical behaviour of suitable soft materials.
    \item Compare the mechanical properties of these materials against the mechanical properties of the human tendons involved in the motion of the knee joint.
    \item Identify the techniques being used for the modelling of the viscoelastic behaviour of soft materials, in applications for human assistance.
    \item Develop a better modelling approach, or optimize current ones, to allow the accurate prediction of the viscoelastic stress response of soft materials.
    \item Assess the performance of the latter modelling approach.
    \item Assess the prediction accuracy of the mentioned modelling approach:
    \begin{itemize}
        \item Design a simulation environment based on a known soft actuator controlling a load.
        \item Establish the control system target based on the torque-speed requirements of the knee joint.
        \item Assess the compatibility of the modelling approach when being implemented as part of the mentioned control system for the real-time prediction of the stress response of a soft material.
    \end{itemize}
\end{itemize}

\section{Scope of the research}

This research investigates the potential benefits of using a viscoelastic material instead of the traditional metallic spring in series-elastic actuators. For this reason, the following materials are investigated: ethylene polypropylene rubber (EPR), fluorocarbon rubber (FR), nitrile rubber (NR), natural rubber with polyester(NatR),  polyethylene  rubber  (PR),  silicone  rubber  (SR), and  100\% natural rubber (NatR100).

On one hand, the research is focused on modelling the complex mechanical behaviour of seven elastomers using a model-driven approach, based on the Linear Viscoelastic Models (LVMs) and a data-driven approach, based on Artificial Neural Networks (ANN). On the other hand, the research investigates the performance of the mentioned approaches when being implemented as part of a control system which is based on a series-elastic actuator model. The scope of this research are as follows:

\begin{itemize}
    \item To characterize the viscoelastic properties of seven rubber-based elastomers using the mechanical tests of stress relaxation and tensile strength, in which different strain rates are included.
    \item To assess the benefits and limitations of modelling the viscoelasticity of soft materials using a model-driven approach, such as, the Linear Viscoelastic Models (LMVs) or a modified version of these.
    \item To assess the benefits of limitations of modelling the viscoelasticity of soft materials using a data-driven approach, such as, different types of Artificial Neural Networks (ANNs).
    \item To assess the performance of the model-driven and data-driven approaches, when predicting the stress-strain curve of the materials under different strain rates.
    \item To design a control system model consisting of a soft series-elastic actuator doing torque-speed control on a single joint, based on the human knee joint requirements during level ground walking (LGW).
    \item To assess the real-time prediction ability of the mentioned approaches when being implemented in a series-elastic actuator model.
\end{itemize}

\section{Contributions of the research}

The research contributes to the field of Soft Robotics for human assistance by investigating the performance of Artificial Neural Networks, as an alternative to the Linear Viscoelastic Models, for the modelling of viscoelasticity in soft materials. Some parts of this thesis are published in peer-reviewed conference papers. The contributions are summarized as follows:

\begin{itemize}
    \item Characterization of the stress relaxation and tensile strength properties, under different strain rates, of seven elastomers. The elastic and viscoelastic properties of the materials are presented in this thesis in the form of tables and plots.
    \item Development of an optimized algorithm to apply the piecewise linearization (PL) method to the Standard Linear Solis model, which allows it to approximate the strain-dependent stiffness of soft materials.
    \item Assessment of the PL method performance, specifically the trade-off between computational cost and accuracy.
    \item Systematic assessment on the performance of different types of artificial neural networks (ANNs) for the prediction of the mechanical behaviour of soft materials. The studied parameters on the developed ANNs are: number of hidden layers, number of neurons, training algorithms, activation functions, optimization algorithm.
    \item Comparison analysis on the ability of the PL method and the ANNs to model the strain-rate dependent stress response of the studied soft materials.
    \item Comparison analysis on the real-time prediction abilities of the PL-method and the ANNs when being implemented as part of a series-elastic actuator model. The control system, developed in Simulink, is tasked to follow the torque-speed requirements of the human knee joint during level ground walking, as a preliminary proof of concept.
\end{itemize}

\section{Outline of the thesis}

The second chapter describes the state of the art on soft robotics applied to human assistance. The latter is organized in three main sections: soft actuation technologies, soft perception systems and soft robotics controllers. However, the main focus of this chapter is directed to innovative soft actuation technologies by describing their advantages, limitations and possible solutions. Furthermore, the design aspects of the soft orthoses and soft exosuits analysed for the realization of this chapter are also described. The design approach of mimicking the human musculoskeletal functionality in soft orthoses, is adopted for this research. The latter means the soft hybrid actuator to be develop will be proximal to the assisted joint and function as an artificial muscle-tendon system. Finally, this literature review highlighted two main research opportunities: the integration of more than one soft actuation technology in the development of a soft hybrid actuator to overcome the individual limitations of each technology; and the implementation of a soft material as an artificial tendon, which replicates the human tendon viscoelasticity properties as close as possible.

The third chapter presents a brief introduction to human biomechanics, putting into con-text the relevant terminology; and describes the characterization process of the human lower limb parameters during ADL, with the aim of obtaining design guidelines for the soft hybrid actuator to be developed. Four main ADL are investigated from clinical trials: ground level walking, going up/down steps, going up/down a ramp and sitting down/standing up from a chair. In a similar way, three mechanical parameters are extracted: torque, power and angle, for the hip, knee and ankle joint. The compiled data is post-processed to obtain a graphical representation and facilitate the comparison and analysis of the parameters variations from subject to subject and activity to activity.

The fourth chapter explains the preliminary experimental work performed with the aim of finding a soft material able to replicate as close as possible the mechanical properties of the human tendon. The latter is motivated by the commonly practice of assuming a perfectly stiff tendon when implemented the mechanical model of the human muscle-tendon system in soft orthoses. Six different types of composite materials, based on rubber and polymers, are se-lected as possible candidates. As a preliminary stage, mechanical tests for measuring both elastic and time-dependant properties of the material were performed to one of the six differ-ent materials, polyethylene rubber. Subsequently, the data is post-processed and compared with the properties found in human tendons. The results highlighted the highly different elas-tic properties between materials, on the other hand, both exhibit similar time-dependant me-chanical properties.

The fifth chapter presents the achieved conclusions, future work and proposed research plan.