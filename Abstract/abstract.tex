
% Thesis Abstract -----------------------------------------------------


%\begin{abstractslong}    %uncommenting this line, gives a different abstract heading
\begin{abstracts}        %this creates the heading for the abstract page

\setlength{\parindent}{17.62482pt}
\setlength{\parskip}{0.0pt plus 1.0pt}

There is an increasing interest in translating the functionality of the human musculoskeletal system (HMS) into soft robotics applications. Soft materials, such as elastomers, have viscoelastic properties similar to the HMS ones. Reliable modelling tools capable of predicting this viscoelasticity are required. Current modelling tools rely on extensive and complex calculations to achieve good accuracy, making them incapable to be deployed in control systems due to a limited computational power. There are two alternatives to circumvent this limitation. The first one is simplifying or linearizing model-driven approaches, i.e. sacrificing accuracy in favor of reducing the model complexity. The second one is using data-driven approaches such as  Artificial Neural Networks (ANNs). This increases the design and developing stage in favor of having high accuracy. 

The aim of this research was, on the one hand, the piecewise linearization (PL) method applied to the Standard Linear Solid (SLS) model is investigated and optimised. On the other hand, the effect of the selection of inputs/outputs, number of neurons in the hidden layer, type of activation function, and training algorithm, on the performance of ANNs is investigated. The tensile strength, under different strain rates, and the stress relaxation of seven rubber-based elastomers are extracted to assess the prediction performance of both approaches. The results of the analysis showed an exponential relationship between the complexity of PL-SLS model and its accuracy. Also, this model was unsuccessful in modelling the strain-rate dependency of the materials. In contrast, most of the proposed ANNs could achieve values of $R^2>0.9$, and low mean square errors (MSE). The Log-Sigmoid activation function has the lowest performance values in both cases. Moreover, there was no substantial difference between the performance of the Bayesian Regularization and the Levenberg-Marquardt training algorithms. The performance of the ANNs did not improve in proportion to the number of neurons. The ANNs real-time prediction was assessed and compared, against the PL-SLS model, in Simulink. Lastly, having the strain-rate of the material as input to the ANNs greatly improves the accuracy, at the cost of adding instability to the control system. Simulation times for the PL-SLS model were much higher in comparison to ANNs.

%The best ANNs candidates were selected based on three main aspects: performance (mse) during train, validation and test, $R^2$ values across all different strain rates, and performance (mse) when presenting the ANNs with unknown experimental data. Subsequently, their real-time prediction was assessed and compared, against the PL-SLS model, in Simulink. In conclusion, this research presents a detailed description on the design, development and implementation of model-driven and data-driven approaches for the prediction of the viscoelastic mechanical behaviour of soft materials, and assess the performance of these approaches when being implemented as part of a control system in a robotic application.

\end{abstracts}
%\end{abstractslong}

% ----------------------------------------------------------------------

%%% Local Variables: 
%%% mode: latex
%%% TeX-master: "../thesis"
%%% End: 
