
% Thesis Abstract -----------------------------------------------------


%\begin{abstractslong}    %uncommenting this line, gives a different abstract heading
\begin{abstracts}        %this creates the heading for the abstract page

%\setlength{\parindent}{17.62482pt}
%\setlength{\parskip}{0.0pt plus 1.0pt}

There is an increasing interest in translating the functionality of the human musculoskeletal system (HMS) into soft robotics applications. Soft materials, such as elastomers, have viscoelastic properties similar to the HMS ones. Reliable modelling tools capable of predicting this viscoelasticity are required. Current modelling tools rely on extensive and complex calculations to achieve good accuracy, making them incapable to be deployed in control systems where computational power is limited. There are two alternatives to circumvent this limitation. The first one is simplifying or linearizing model-driven approaches, i.e. sacrificing accuracy in favor of reducing the model complexity. The second one is using data-driven approaches such as  Artificial Neural Networks (ANNs). This increases the design and developing stage in favor of having high accuracy. 

The aim of this research is to systematically compare the performance of the two mentioned approaches when being used, as part of a control system, to predict the mechanical behaviour of seven different elastomers. On the model-driven side, the piecewise linearization (PL) method applied to the Standard Linear Solid (SLS) model is investigated. On the data-driven side, different topologies of Feedforward Neural Networks (FFNNs) are investigated. In both cases, the impact on the selection of relevant parameters is studied. The tensile strength, under different strain rates, and the stress relaxation characteristics of the selected elastomers are extracted and used to assess the prediction capabilities of both approaches. 

%(THIS IS NOT FINISHED) The results of this research characterized the exponential relationship between the complexity of PL-SLS model and its accuracy. Also, this model was not capable of accounting for the strain-rate dependent stress response of the materials. In contrast, most of the proposed ANNs achieved low mean square errors (MSE). The Log-Sigmoid activation function was found to be incompatible with this application. The main difference between the investigated training algorithms is visible in the $R^2$ plots. The Bayesian Regularization. The performance of the ANNs did not improve in proportion to the number of neurons. The ANNs real-time prediction was compared against the PL-SLS model in Simulink. Lastly, having the strain-rate of the material as input to the ANNs greatly improves the accuracy, at the cost of adding instability to the control system. Simulation times for the PL-SLS model were much higher in comparison to ANNs.

%The best ANNs candidates were selected based on three main aspects: performance (mse) during train, validation and test, $R^2$ values across all different strain rates, and performance (mse) when presenting the ANNs with unknown experimental data. Subsequently, their real-time prediction was assessed and compared, against the PL-SLS model, in Simulink. In conclusion, this research presents a detailed description on the design, development and implementation of model-driven and data-driven approaches for the prediction of the viscoelastic mechanical behaviour of soft materials, and assess the performance of these approaches when being implemented as part of a control system in a robotic application.

\end{abstracts}
%\end{abstractslong}

% ----------------------------------------------------------------------

%%% Local Variables: 
%%% mode: latex
%%% TeX-master: "../thesis"
%%% End: 
