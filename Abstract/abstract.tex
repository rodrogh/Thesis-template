
% Thesis Abstract -----------------------------------------------------


%\begin{abstractslong}    %uncommenting this line, gives a different abstract heading
\begin{abstracts}        %this creates the heading for the abstract page

\setlength{\parindent}{17.62482pt}
\setlength{\parskip}{0.0pt plus 1.0pt}
	
This emerging field of soft robotics is currently being applied for human assistance applications such as soft orthoses or soft exosuits able to assist human movements. The elder people is main sector of the population which will be directly benefited by the development of a portable, compliant, lightweight and efficient soft exosuit. Current developments rely on well established technologies such as pneumatic artificial muscles and electric motors paired with Bowden cables. However, many research is aiming on improving both the established technologies and emerging technologies such as shape memory materials. The current report describes the state of the mentioned field of interest while paying particular attention to the soft actuation technologies advantages, limitations and proposed improvements. In a similar way, the soft perception technology is as well discussed along the control techniques being implement-ed in soft exosuits and soft orthoses. Furthermore, the development of a hybrid soft actuator is proposed by combining two or more available technologies in an attempt to overcome their individual limitations. Moreover, as a preliminary design step, it is described the characterization of the parameters for the human lower limb for daily living activities which aims to provide the information required to tailor the hybrid soft actuator design parameters. The characterization considers four main daily activities: walking, using steps, walking on a ramp and using a chair. In the same way, the parameters of torque, angle and power during such activities were recorded from several clinical trials and plotted for analyzing. Through the obtained charts it is possible to define the required parameters of the soft hybrid actuator to be designed. Finally, as a preliminary experimental work, the feasibility of imitating the musculoskeletal-tendon system is assessed by characterizing the mechanical properties of polyethylene rubber, since the literature suggest that this type of material have similar properties as the human tendon.

\end{abstracts}
%\end{abstractlongs}

% ----------------------------------------------------------------------

%%% Local Variables: 
%%% mode: latex
%%% TeX-master: "../thesis"
%%% End: 
